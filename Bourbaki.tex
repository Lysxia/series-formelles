\documentclass{article}

\usepackage{url}
\usepackage{amsmath}
\usepackage{amsfonts}
\usepackage{amssymb}

\newcommand{\C}{\mathcal{C}}

\begin{document}

\noindent
Bourbaki, \emph{Livre I Th\'eorie des ensembles Chap.IV (\'etat 7?)
  Structures}, p. 6.
\url{http://sites.mathdoc.fr/archives-bourbaki/PDF/180_nbr_083.pdf}

\section*{Esp\`eces de structures et structures.}

On dit qu'on a d\'efini dans une th\'eorie $\C$ plus forte que la
th\'eorie des ensembles une \emph{esp\`ece de structure}, lorsqu'on
s'est donn\'e:
\begin{enumerate}
\item Un certain nombre de lettres distinctes $x_1, \dots, x_n, s_1,
  \dots, s_p$ autres que les constantes de $\C$.
\item Des \'echelons $T_1, T_2, \dots, T_p$ d'une construction
  d'\'echelons (1) sur les $n$ lettres $x_1, \dots, x_n$, en nombre
  \'egal \`a celui des lettres $s_j$, distincts ou non.
\item Une relation $R\{x_1, \dots, x_n, s_1, \dots, s_p\}$ de la
  th\'eorie $\C$, de la forme
  \begin{multline*}
    s_1 \subset T_1(x_1, \dots, x_n) \text{ et } s_2 \subset T_2(x_1,
    \dots, x_n) \text{ et \dots et } \\
    s_p \subset T_p(x_1, \dots, x_n) \text{ et } R'\{x_1, \dots, x_n,
    s_1, \dots, s_p\}
  \end{multline*}
de fa{\c c}on que la relation suivante soit un th\'eor\`eme de $\C$:

(IS) $(R\{x_1, \dots, x_n, s_1, \dots, s_p\}$ et $f_1$ est une
bijection de $x_1$ sur $y_1$ et \dots et $f_n$ est une bijection de
$x_n$ sur $y_n) \implies R\{y_1, \dots, y_n, s_1', \dots s_p'\}$ ou
$y_1, \dots, y_n$ sont des lettres distinctes des constantes de $\C$
et de toutes les lettres figurant dans $R\{x_1, \dots, x_n, s_1,
\dots, s_p\}$, et ou on a pos\'e \[ s_j' = T_j\langle f_1, \dots,
f_n\rangle \langle s_j \rangle \] pour $1 \leq j \leq p$.

La d\'efinition des termes $T_j\langle f_1, \dots, f_n\rangle$ se fait
bien entendu dahs la th\'eorie obtenue en adjoignant \`a $\C$, d'une
part l'axiome ``$f_1$ est une bijection de $x_1$ sur $y_1$ et \dots et
$f_n$ est une bijection de $x_n$ sur $y_n$'', et d'autre part l'axiome
``$s_1 \subset T_1(x_1, \dots, x_n)$ et \dots et $s_p \subset T_p(x_1,
\dots, x_n)$''.

\section*{Species of structures and structures.}

We say that we have defined, in a theory
$\C$ at least as strong as set theory, a \emph{species of structure}
when we have:
\begin{enumerate}
\item A certain number of distinct variables $x_1, \dots, x_n, s_1,
  \dots, s_p$ besides the constants of $\C$.
\item
\item A relation $R\{x_1, \dots, x_n, s_1, \dots, s_p\}$ of the theory
  $\C$, of the form
  \begin{multline*}
    s_1 \subset T_1(x_1, \dots, x_n) \text{ and } s_2 \subset T_2(x_1,
    \dots, x_n) \text{ and \dots and } \\
    s_p \subset T_p(x_1, \dots, x_n) \text{ and } R'\{x_1, \dots, x_n,
    s_1, \dots, s_p\}
  \end{multline*}
\end{enumerate}
\end{enumerate}


\end{document}
